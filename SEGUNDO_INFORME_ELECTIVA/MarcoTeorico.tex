\section{Marco Te�rico} \label{sec:marcoteorico}

Como medio de adquisici�n de datos se uso la tarjeta STM NUCLEO411R la cual se encargo realizar la tarea de adquisicion de datos enviados por la MPU a traves de comunicacion i2c y enviar estos datos por media de una comunicacion As�ncrona por puerto serial a la computadora, estos datos fueron recibidos en la computadora por la aplicaci�n Matlab la cual fue la encargada de realizar el procesamiento de los datos con el cual se grafican los datos y se puede observar un comportamiento frente a los datos despu�s de ser calibrados correctamente.

\subsection{Elementos Utilizados}
Se utilizo cuatro elementos f�sicos para la realizaci�n de la practica 


\subsubsection{MPU6050}

EL MPU6050 es una unidad de medici�n inercial o IMU (Inertial Measurment Units) de 6 grados de libertad (DoF) pues combina un aceler�metro de 3 ejes y un giroscopio de 3 ejes. Este sensor es muy utilizado en navegaci�n, goniometr�a, estabilizaci�n, etc.\

\includegraphics[scale=0.2]{../../Downloads/MPU6050.jpeg} \cite{amarante2008questao}

\subsubsection{NUCLEOF411RE}

La tarjeta de desarrollo STM32 provee una soluci�n accesible y flexible para el desarrollo de prototipos y nuevas ideas bajo una plataforma ARM Cortex-M4 de ST y la plataforma mbed ARM. La tarjeta se dise�o bajo caracter�sticas tanto en el microcontrolador como en sus componentes de un sistema balanceado en potencia de procesamiento con un consumo de energ�a limitado. La tarjeta es compatible en pines con un Arduino y con ST Morpho.\

\includegraphics[scale=0.2]{../../Downloads/NUCLEOF411RE.jpeg} \cite{amarante2008questao}

\subsubsection{Computadora Port�til}

computadora port�til de peso y tama�o ligero, su tama�o es aproximado al de un portafolio (hay m�s peque�as como Palmtop y Handheld). �sta pertenece al grupo de las computadoras personales, las cuales son sistemas de computaci�n relativamente peque�os y de bajo costo, tambi�n llamados microprocesadores.\

\includegraphics[scale=0.18]{../../Downloads/PC.jpeg} \cite{amarante2008questao}\
 
\subsubsection{Programaci�n utilizada}

La programaci�n utilizada fue brindada por el docente de la clase Prof. Msc. Fabi�n Barrera Prieto.
Primero utilizamos un programa de calibracion mediante la cual adquirimos dos muestras de 100 datos de las cuales la primera muestra es descartada como dato estad�stico, de la segunda muestra de datos son tomados para analisiz y obtener los datos de inter�s para la practica los cuales son los offset en cada uno de los ejes de los dos sensores, estos datos son obtenidos a trav�s de condicionales aplicadas a los puntos m�ximos y m�nimos de los valores de cada uno de los ejes, estos valores son promediados para obtener un offset por cada eje.
A continuaci�n podemos encontrar el link al c�digo utilizado para la calibraci�n :

https://drive.google.com/file/d/
1Z_q4SNt6_mHB1RLBcFIv_
cTe6h7G8vUV
/view

Para la parte de lectura de datos utilizamos un codigo en el cual se realizaba la lectura del puerto i2c de la tarjeta NUCLEO, esta programaci�n realiza la lectura cuando a trav�s del puerto serial recibe un car�cter 'H', posteriormente  escribimos al registro 3B de la IMU, despu�s realizamos la lectura de la IMU a trav�s de i2c de 14 Bytes y almacenando los datos recibidos en un vector de 14 posiciones,luego de recolectados estos datos son enviados a trav�s de el puerto serial a la computadora.

A continuaci�n podemos encontrar el link al c�digo utilizado para la calibraci�n :

https://drive.google.com/file/d/1LdlBcjZGbrdfhYsP65o36baeUx
7m4zoM/view

Para el procesamiento de datos utilizamos Matlab 
mediante este procesamiento realizamos una lectura de los datos enviados por la STM411R, estos datos son calibrados y mostrados a trav�s de gr�ficas de los valores de los ejes sin calibrar y calibrados 

A continuaci�n podemos encontrar la programaci�n utilizada en Matlab:\\\

close all;\\
clear all;\\
clc;\

oldobj = instrfind;\\
if not(isempty(oldobj))\\
    fclose(oldobj);\\
    delete (oldobj);\\
end
\\
if ~exist('s','var')\\
    s = serial('COM5','BaudRate',9600,'DataBits',8,'Parity','None','StopBits',1);\\
end\\
if ~isvalid(s)\\
    s = serial('COM5','BaudRate',9600,'DataBits',8,'Parity','None','StopBits',1);\\
end\\
if strcmp(get(s, 'status'),'closed')\\
    fopen(s);\\
end\\

SENSITIVITY_ACCEL = 2.0/32768.0;\\
SENSITIVITY_GYRO = 250.0/32768.0;\\
offset_accelx = 120.00;\\
offset_accely = 68.00;\\
offset_accelz = 16932.00;\\
offset_gyrox = 4053.00;\\
offset_gyroy = 111.00;\\
offset_gyroz = -124.00;\\

disp('En sus marcas. Posicione el sensor en la posicion inicial')\\
pause();\\

disp('comienza')\\
fprintf(s, 'H');\\
i = 1;\\
while(1)\\
    str{i} = fscanf(s);\\
    if(str{i}(1) == 'A')\\
        disp('termina')\\
        break;\\
    end\\
    i = i + 1;\\
end\\

fclose(s);\\
n = length(str)-1;\\
c = 1;\\

for i=1:n\\
    temp = cellfun(@str2num,strsplit(str{i},','));\\
    if nume1(temp) == 8\\
        values(i,:) = temp;\\
    end\\
end\\

save DadosTest1 values\\

Nsamples  = length(values);\\
dt = 0.01;\\
t = 0:dt:Nsamples*dt-dt;\\

figure;\\
plot(t, values(:,3)*SENSITIVITY_ACCEL, 'b')\\
hold on\\
plot(t, values(:,4)*SENSITIVITY_ACCEL, 'r')\\
plot(t, values(:,5)*SENSITIVITY_ACCEL, 'g')\\
title('Acelerometro sin calibrar')\\
xlabel('Tiempo (segundos)')\\
ylabel('aceleracion (g)')\\
legend('ax', 'ay', 'az', 'Location','northeast', 'orientation','horizontal')\\


figure;\\
plot(t, (values(:,3)-offset_accelx)*SENSITIVITY_ACCEL, 'b')\\
hold on\\
plot(t, (values(:,4)-offset_accely)*SENSITIVITY_ACCEL, 'r')\\
plot(t, (values(:,5)-(offset_accelz-(32768/2)))*SENSITIVITY_ACCEL,'g')\\
title('Acelerometro calibrado')\\
xlabel('Tiempo (segundos)')\\
ylabel('aceleracion (g)')\\
legend('ax', 'ay', 'az', 'Location','northeast', 'orientation','horizontal')\\


figure;\\
plot(t, values(:,6)*SENSITIVITY_GYRO, 'b')\\
hold on\\
plot(t, values(:,7)*SENSITIVITY_GYRO, 'r')\\
plot(t, values(:,8)*SENSITIVITY_GYRO, 'g')\\
title('Giroscopio sin calibrar')\\
xlabel('Tiempo (segundos)')\\
ylabel('velocidad angular')\\
legend('gx', 'gy', 'gz', 'Location','northeast', 'orientation','horizontal')\\

figure;\\
plot(t, (values(:,6)-offset_gyrox)*SENSITIVITY_GYRO, 'b')\\
hold on\\
plot(t, (values(:,7)-offset_gyroy)*SENSITIVITY_GYRO, 'r')\\
plot(t, (values(:,8)-offset_gyroz)*SENSITIVITY_GYRO, 'g')\\
title('Giroscopio calibrado')\\
xlabel('Tiempo (segundos)')\\
ylabel('velocidad angular')\\
legend('gx', 'gy', 'gz', 'Location','northeast', 'orientation','horizontal')\\



